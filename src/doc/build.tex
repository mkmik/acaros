\chapter{Build system}
\label{ch:build}

\section{Introduction}

\acaros attempts to have a clean, easy to use and \emph{template free} build system.
That means that unlike \emph{autoconf/automake}-like systems, the Makefiles are not
dynamically generated, but instead they are a handful of very short and modular makefile scripts
which are included from a 3-4 line makefile in every directory. 

The makefile in most directories need not to be changed, even when adding source files,
so simplifying the version control and branch merging.

\section{Version control}

\textsf{Subversion}\footnote{http://svnbook.red-bean.com/svnbook/} was chose to host the project, because \ldots you will find out using it.

Repository access requires SSH access for now.
The repository path is:

\begin{verbatim}
  svn+ssh://sottosale.net/home/users/kaos/repos/acaros/trunk/
\end{verbatim}

Basic subversion commands description assumes the \textsf{REPOS} to be set
to the value of the repository path shown above.

\paragraph{Checking out}
\begin{verbatim}
  svn co $REPOS acaros
\end{verbatim} % $

This will create a subdirectory called \textsf{acaros} the current directory.
Obviously you could named as you like.

This directory will be a \emph{working dir}.

\paragraph{Updating}
\begin{verbatim}
  svn update
  svn up
\end{verbatim}

Will fetch new releases from the server.

\paragraph{Getting status}
\begin{verbatim}
  svn status
  svn st
\end{verbatim}

Will list in what state the files in the in the working dir in respect of
the lastly updated revision in the repository''

\begin{description}
\item[?] not versioned
\item[!] missing
\item[M] locally modified
\item[D] locally deleted
\item[A] locally added
\item[L] locked
\end{description}

\ldots

\paragraph{Adding files and directories}
\begin{verbatim}
  svn add (file|dir)*
  svn add myfile.c mydir
\end{verbatim}

\paragraph{Renaming files and directories}
\begin{verbatim}
  svn mv oldName newName
  svn mv blabla/dir1 there/dirzz
\end{verbatim}

\textbf{Major feature}. This feature is the most important improvement over CVS.

\paragraph{Ignore some files}
\begin{verbatim}
  svn propedit svn:ignore dir1
  svn propget svn:ignore dir1 | 
    svn propset svn:ignore -F /dev/stdin dir2
\end{verbatim}

Allows to specify filters (like *.a) to compilation products or other files
that you don't want to be versioned (or to produce the annoying ``?'' in \textsf{svn status})

\section{Makefiles}

Most makefiles (except src/Makefile, src/kernel/Makefile, \ldots) are very short and duplicate
the bare minimum.

The only part which changes is list of subdirectories which will be recursively traversen in the
build phase.

This subdirectories are called \emph{subsystems} and defined in the \textsf{SUBSYSTEM-y} variable.
Makefiles need only to append the named of the subdirectories that should be processed
according to the current configuration. This is done using the following trick:

\begin{verbatim}
  SUBSYSTEMS-y += log mm
  SUBSYSTEMS-$(CONFIG_KDB) += kdb  
\end{verbatim} % $
 
\noindent In this example the subsystems \texttt{log} and \texttt{mm} are forcefully enable 
while the subsystem \texttt{kdb} will be enabled only if the variable \texttt{CONFIG\_KDB}
evaulates to ``\texttt{y}''.\footnote{defined in \textsf{.config} if enabled from menu}


\section{Configuration}

The \acaros compilation can be tailored with optional parameters using a ncurses menu driven interface, using a simple \texttt{make config} command.

The program that handles this configuration was ``stolen'' from the linux build system.

It basically works parsing the \textsf{Kconfig} files and generating a \textsf{.config} and 
an \textsf{include/autoconf.h} file. \textsf{.config} will be included in every makefile
while \textsf{autoconf.h} will be include by source files if the implement some feature 
conditionally.

Makefiles can conditionally include subdirectories in the subsystem path for the module;
this way for example the kernel debugger will not even be compiled if you don't enable it.

\section{Build}

A simple \texttt{make} invocation should build the whole kernel according to current configuration.

The default \textsf{make} target doesn't include a dependency tree generation because it's slow
and is left to manual use. So unless you don't issue a \texttt{make deps} modifying a header
file will not recompile the dependent C files. 

Some times, when the dependency is not enough
and a complete rebuild is needed, a handy command \texttt{make rebuild} does the job for you,
a complete clean,deps,make,install pass.

The build system build the two main components of the system
leaving them at their original locations (\textsf{arch/\${ARCH}/loader/loader} and
\textsf{kernel/kernel} \textsf{kernel/acaros.tar.gz}. The kernel image is encapsulated
in a tar archive with other boot modules and drivers. On boot this tar archive is opened
by the loader which locates the kernel and loads it.\footnote{This way the multiboot
command line can stay short.}

\section{Installation}

Since \acaros development cycle is intended to be fast and with extensive testing
of every line of code\footnote{a sort of extreme programming but less extremist}, 
there must be a way to quikly install and boot the target system.

\subsection{Emulators}
One way to quickly test \acaros is to use an emulator like \textsf{bochs} or \textsf{vmware},
\subsubsection{Vmware}
Works. Network + vesa too.

The build system assumes you have your virtual machine in \textsf{~/vmware/acaros}
\footnote{Actually instead of \textsf{acaros} it searches the current branch name,
that is the name of the toplevel directory (the one below \textsf{src}).},
and it's important for serial port redirection, used mostly by the kernel debugger.

Important vmware configuration are:
\begin{description}
\item[Memory] choose whatever you want. \acaros assumes >1.5, but just for now.
\end{description}

\subsubsection{Bochs}

Works but hangs on kernel debugger.

The \textsf{.bochsrc} file is provided. 

It boots only using the floppy image.

\subsubsection{qemu}

It doesn't work with \acaros \ldots 

\subsection{Booting \acaros}

There are basically two ways quickly prepare to boot \acaros:
\begin{itemize}
\item floppy images
\item network booting
\end{itemize}

\texttt{make install} will automatically reckognize which boot system is enabled
\footnote{By looking if the floppy image is mounted, or a tftp directory is present} and
install the kernel in the right places. 

This magic needs some collaboration however, as explained in the next sections:

\subsubsection{Floppy image}

You need a floppy image with grub installed on it. You can download one 
from http://www.acari.org/download/acaros.img 

Once you put this file in the root of the build system you can now
choose to mount it manually or using the \texttt{make loop\_mount} command
which exploits the \texttt{sudo} command, and mounts the image in loopback for you.

The image should be mounted in the \textsf{image} directory so that \texttt{make install}
can find it.

The \textsf{support} directory contains an example grub configuration file.

\subsubsection{Network boot}

\texttt{make install} searches a \textsf{/home/tftp} directory and if it exists
it copies the kernel and the loader in it. 

Since grub is needed to boot the kernel you need to compile \textsf{pxegrub} and 
configure \textsf{dhcp} and \textsf{in.tftpd} to run it for the MAC associated
with your testing machine.\footnote{real or emulated (vmware can boot from pxe as of version 4.5)}

The \textsf{support} directory contains an example grub configuration file for network boot.

\section{Debugging}

The kernel is by default compiled with all debugging symbols enabled.

The build system helps also debugging by implementing a \textsf{make debug} command
which installs the kernel, spawns a \emph{gdbredirector} and a gdb instance.

You need only to attach to the tcp port 1235, if you use VMware, with:

\begin{verbatim}
  target remote :1235
\end{verbatim}

\noindent or directly to the serial port

\begin{verbatim}
  target remote /dev/ttyS0
\end{verbatim}

\noindent and you will attach a running kernel through the serial port.

VMware can bind an emulated serial port to an unix domain socket (which he calls incorrectly \emph{pipe}). The \texttt{gdbredirector} program converts TCP/IP from gdb to unix domain socket on the vmware side.\footnote{It can also be used for analyzing the gdb protocol}

\section{Documentation}

This documentation is located in \textsf{src/doc} and is builded with a simple \textsf{make}.
The reference manual is build using \textsf{make api}, and it outputs \textsf{api/html} and \textsf{api/latex}. If you want to generate the pdf reference manual you should do another \textsf{make} inside \textsf{src/doc/api/latex}.

\section{Implementation}

TODO \ldots

