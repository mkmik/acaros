\chapter{Kernel}

\section{Introduction}

\section{Building}

The kernel is built as a normal ELF static executable without standard libraries. 
A linux compiler can produce suitable format and code. 

The kernel is linked to run in the upper virtual memory area (should be configurable
but now at 0xe0000000, 3.5 GB). The loader reads the addresses in the ELF header (which can be inspected with \textsf{objdump} and allocates and copies the kernel text and data segments in the allocated
virtual memory regions. 


\section{Boot}

The kernel, when executed, runs directly with virtual memory enabled.\footnote{Set up by the loader
or on some architecture by the boot firmware}

The kernel entry point is it's \textsf{\_start} function.
The loader passes a structure that describes the arguments, the physical and virtual memory
maps, the loaded modules and other things.\footnote{Vesa informations, for example on PCs.}

Then vital builtin modules are initialized (\textsf{mal}, 
\textsf{kdb}\footnote{Optional.}, \textsf{mm}, \textsf{co}, \textsf{drivers}\footnote{in \textsf{ultrabug branch}}) and then the kernels friendly panics.

That's all folks :-)

\section{KDB}

The kernel debugger is a little and tricky part of the kernel. It implements the \textsf{gdb}
serial protocol and implements stopping, resuming, breakpointing, single stepping, memory and register reading and writing.

Using this simple primitives \textsf{gdb} can implement full source level debugging.

I/O through the serial is handled using a builting driver inside the kdb in order to not depend on the rest of the kernel. I/O is interrupt driven and character based. 

See the info page for gdb under ``* Remote Protocol::'' for detailed description of the protocol.

One of the design rules of \textsf{kdb} was that it must use a fixed amount of memory
and be able to handle big requests without allocating memory for buffers.

\emph{So it was implemented using a state machine that accept characters at imput and depending
on the character changes the state. This is very good for parsing the packet,
 parsing hexadecimal addresses but showed to be quite difficult to maintain design rule
if gdb askes us to write big amounts of memory, and passes a big stream of data, and
of course if gdb doesn't ask us to do so, then what's about implementing it as a state machine at all? pure mental masturbation?}

\subsection{Interruption, resumption}

Since the reception of characters is interrupt driven we can interrupt the kernel 
whatever it is doing if we got a special condition:

\begin{enumerate}
\item First time the debugger sent a valid packet (debugger is attacching)
\item Debugger sent character 0x3 (INT)
\item A breakpoint exception occour (int 3)
\item A single step condition occour (int 1)
\end{enumerate}

\noindent In order to block the kernel indefinitely in this cases, kdb simply never returns
from the interrupt handler but spins on a global variable waiting it to change state.

Resetting this variable the interrupted kernel will continue to execute as nothing happens
because the serial port interrupt or breakpoint interrupt has saved all the state and will restore it upon return.

This loop can be interrupted only by other serial port interrupts which eventually
can resume the execution later on.

\subsection{Program state}

Gdb needs to know the current state of the machine registers every time the target stops.

Kdb can send the content of the registers ``on demand'' responding to the ``\texttt{G}'' gdb command
or it can feed gdb with the information of the most important registers directly when reporting
that software is stopped, with the ``\texttt{T}'' packet (see gdb manual).

In any case it must access the state saved on interrupt. \textsf{mal} handles this
very conveniently using a short assembler wrapper in front of every interrupt handler
which pushes on the stack all the registers so that you can access them from C using
\textsf{mal\_trapFrame} (on x86). Modifing this registers has a real effect of modifying
the registers of the real program once resumed. For example single step is enabled
updating the saved \textsf{eflags} register image setting the \textsf{trace} bit.


\subsection{Tricks \& details}

When executing an breakpoint instruction, on resume the CPU will issue again the
same instruction. This is correct, because normally gdb will handle this situation
deleting the breakpoint and resetting it later, but if the breakpoint is generated
inside the kernel, for example by a panic or manual debugging, before returning
from the breakpoint interrupt we must check whether the breakpoint instruction is still
there and if it is it means that gdb has not set this break point, so we must skip it.
(incrementing the instruction pointer)

\subsection{Performance issues}

\todo

Reading and writing is done using byte by byte page protection checking and this could be optimized

