\chapter{Introduction}

\acaros\ is an operating system kernel targeted for workstation and server class
machines. The design permits easy ports to embedded class machines but this
document will focus only on the implementation on mid/high range machines
which include MMU (virtual memory), privilege separation (kernel/user),
external bus (PCI, \ldots), DMA, and programmable interrupt controllers.

\acaros will be the kernel of the \kaos project which will build an complete system
based on \acaros.

\section{Design notes}

\acaros\ does not follow any schoolbook design. It is not a microkernel
design but neither a monolithic UNIX style kernel. \acaros design
goal is to merge microkernel concepts with performance needs
allowing processes to run inside kernel space or user space using basically
the same API, thus allowing easier developing and debugging in user space
and increased security for untested drivers or extensions\footnote{expecially third
party}, while maintaining breakthrough performance for critical path code
like server networking and filesystem operation. 

The design allows an administrator (or an end user though the installer program)
to choose the level of protection vs. performance, moving in and out modules from
user to kernel. 

As stated before this is the design goal, not necessarly the goal for the first implementation,
but many design decisions are influenced by that.

Other design decision is to clearly protocolize intra-module and intra-layer
communication reducing the binary incompatibility between different releases/builds
\footnote{Like linux, most notably}. Object oriented message passing is choosen
to implement this abstraction.
 
We have developed a very simple and limited object oriented extension to 
the C language: CO\footnote{The acronym is open to free interpretation, from \emph{C object}
to \emph{Che cazzO}}.
 It is similar to Objective C but with less emphasis on smalltalk emulation
and with far simpler implementation and with less features. The lack of advanced OO programming
techniques available to the kernel programmer should help him avoid the
abuse of OO in time critical applications such the kernel and keep clear
the releationship between objects, methods and backing memory.
The CO compiler\footnote{coc} is implemented as a preprocessor to C.

No existing object oriented programming language was well suited for kernel
core development, partially because it we found difficulties porting the language runtime
without basic OS support and also we would loose compatibility with the mainstream
compiler at each new version. We didn't want force users to use a specific version
of the C compiler nor special build tools or compilers.
 For now we only assume the GNU tool chain and use some of its peculiarities
 but it the future a more compiler agnostic approach will be tried.\footnote{Is not so 
easy to be compiler agnostic without the use of too much conditional compilation\ldots}

GNU C was choosen for infinite reasions. Portability is granted by
the widespread use of the GNU C Compiler and its infinite incarnations
for practically every known CPU. Explicit use of GNU C extensions
was choosen above macros with compiler specific code to avoid preprocessor
junk and improve readability.

\section{Virtual memory}
\acaros\ kernel uses paged virtual memory for all memory management
and does not direct map physical memory as other OS do\footnote{Linux
maps all physical memory in the kernel address space. The consequence
of this design is that the amount of physical memory is restricted
to 1GB/2GB/2GB (depends on kernel configuration: more address space
to user mode processes, less physical memory available). Recent patches
relax this limitation at cost of performance and code readability.}.

The virtual memory management is conceptually similar to those from
\windows\ NT and \solaris, but is not meant to be compatible with them.

\acaros\ is a modern OS and thus implements the common memory management
functinos like memory sharing, copy-on-write, memory mapped files, device memory mapping,
demand paging, shared mem IPC and zero copy IO (networking).

Platform dependent page structure handling is done throgh a dedicated
HAT (Hardware Address Translation) layer, part of the MAL (Machine Abstraction Layer)
\footnote{from an latin stem which means \emph{evil} with an italian acception of \emph{pain}, which is very well suited in hardware related stuff\ldots}.
 Platform independent code does not access directly page structures through
macros like linux for example, thus it is not necessary to emulate a three level
page structure like it does. 

\section{Bootstrap}

The execution of \acaros\ is divided basically in two parts:

\begin{enumerate}
\item loading kernel
\item executing kernel
\end{enumerate}

\noindent it seems obvious but it depends on what we mean for ``loading''
the kernel:

The kernel itself uses memory resources for its code, data and stack
but the memory management (and paging) is not yet initialized when the kernel is first run.
 One approach would be to put a careful written assembly and C code which prepares 
memory management structures and initialize paging, in order to pass the execution
to the kernel once its living in the right address space\footnote{More information about that at
 \ref{booting-strategies}}.

\acaros\ instead uses an intelligent loader which loads the kernel at the
right virtual addresses contained in its executable image\footnote{Currently ELF is choosen}, 
like any other program, initializing paging and keeping track of each physical and virtual
memory allocation and passing a structure describing the memory layout to the kernel.
 The kernel can then go on and still use loader memory management functions 
for initial memory allocation (like allocation of the Page Frame Database) and then
use this information to fill its memory management structures to reflect the initial allocation.
 After that the initial the full featured memory management routines can be used.
 
This strategy avoid knowing the layout of the kernel memory management structures
when initializing the machine and improves development rates and platform independecy,
at the cost, however, of a little more work and some mental confusion.

The loader itself is platform dependent but has a common protocol with the 
kernel entry point and the kernel memory manager subsystem. It is like a
little OS with limited knowledge of memory management and executable loading. 
Eventually the loader stage could also be able to load the kernel image
and inital modules from disk or it could delegate this to stage 0 loaders
like GRUB.
 Every architecture has it's separation of tasks, involving potentially other entities
interacting before the kernel itself boots, for example on the PC architecture the loader 
interacts with the a multiboot compliant boot loader\footnote{Like grub} which is responsable
for loading the loader and the kernel from disk or network, and switching to protected mode
\footnote{this way simulating finally a real 32-bit machine. \acaros has no 16-bit code, not even on bootstrap ! :-)}

The kernel is a normal ELF executable image compatible with the native Linux executable
format, but probably it should work with solaris build tools as well.

\section{Documentation}

Other than this book there is also a reference documentation generated with \textsf{doxygen}
in \textsf{src/doc/api/html} and \textsf{src/doc/api/latex}. 
See building system at page \pageref{ch:build}.

\section{History}

The project begun when in late 2002 my professor of 
\emph{technical computer science} at the SUPSI\footnote{switzerland university of applied sciences} showed us a little embedded real time OS called \textsf{uCos\texttrademark}, which came you needed came with a crappy build system and you needed to patch several files in order to compile, it ran with 
a 16 bit turbo C compiler and it's scheduler didn't allow two threads to have the same priority.
 
While it is surely a good choice in many application, and it's pors to many platforms are certainly useful I really don't understand why I or my university should pay royalities to a thing like this.
 So I wrote a parody of uCos (spelled micro C O.S) and called it mikuCos (spelled m i q c o s),
which quickly had a full blown time round robin scheduler, keyboard, timer, syncronization primitives, and then arrived a vt100 emulator, uart, pc speaker, virtual filesystem, pci, network driver, paging.
The code was so ugly and eventually was so filled with various memory corruptions that it was
wiser to rewrite everything with less hurry, and so \acaros was born.

It started up as a joke mainly because I've found out how simple it was to load a ELF executable
directly from the GRUB boot loader using standard tools and with little or no assembler, so
the I skipped the most annoying part, the bootloader itself, on which people usually get so
frustrated that they loose interest in OS writing (or simply loose their time) \ldots