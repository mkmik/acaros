\chapter{Library helpers}

Acaros supplies a set of useful library functions and data
structures.

\section{libstand}
\section{libalgo}

\subsection{Linked lists}
The doubly linked list implementation used in acaros
is derived from the linux kernel sources. It is designed
for high performance and low overhead. Instead of
manage separate \emph{nodes}, linking information is embedded
in the manged objects.

Example \ref{fig:link1} shows how to create a structure that
can be a node of two separate lists (\texttt{free\_list} and \texttt{busy\_list}).
 All list handling functions operate on \texttt{list\_head\_t} structures;
you should pass always the pointer to your \texttt{list\_head\_t} member.
 Once you have a pointer to a \texttt{list\_head\_t} you can get back 
the pointer to the original object using the \texttt{list\_entry} macro.
 See figures \ref{fig:listexample1} and \ref{fig:listexample2} for an example.

\begin{figure}[h!]
  \begin{verbatim}
    struct data {
      int a;
      int b;
      list_head_t free_list;
      int c;
      list_head_t busy_list;
      int d;
    };
  \end{verbatim}
  \caption{Linked list example.}
  \label{fig:link1}
\end{figure}

\begin{figure}[h!]
  \begin{verbatim}
    list_head_t free_list_head = INIT_LIST_HEAD(free_list_head);
    
    struct data d1,d2,d3;
    
    list_add(&d1.free_list, &free_list_head);
    list_add(&d2.free_list, &free_list_head);
    list_add(&d3.free_list, &d2.free_list);

    // gives: d2->d3->d1

    list_add(&d1.busy_list, &busy_list_head);
    list_add_tail(&d2.busy_list, &busy_list_head);
    list_add_tail(&d3.busy_list, &busy_list_head);

    // gives: d1->d2->d3
  \end{verbatim}
  \caption{Linked list ordering.}
  \label{fig:listexample1}
\end{figure}

\begin{figure}[h!]
  \begin{verbatim}
    list_head_t *pos;
    struct data *current;
    int sum=0;

    list_for_each(pos, &free_list_head) {
      current = list_entry(pos, struct data, free_list);
      sum += current->a;
    }

    list_for_each(pos, &busy_list_head) {
      current = list_entry(pos, struct data, busy_list);
      sum += current->b;
    }
  \end{verbatim}
  \caption{List traversal.}
  \label{fig:listexample2}
\end{figure}

Figure \ref{fig:listexample1} demonstrates how a specific ordering 
can be obtained using \texttt{list\_add} and \texttt{list\_add\_tail}.
 Figure \ref{fig:listexample2} demonstrates list traversal helper macros
\texttt{list\_for\_each} and \texttt{list\_entry}.
In order to obtain the original structure from a \texttt{list\_head\_t}
you can use \texttt{list\_entry} macro, passing the node pointer, the
type of the structure and the name of the \texttt{list\_head\_t} member.

\subsection{AVL tree}

AVL trees are balanced binary search trees which have $O(\log_{2} n)$ complexity.
The AVL tree concepts in \acaros\ is somewhat similar to those of linked lists:
 the node structure (containing left, right, and skew) is embedded inside the 
custom data structure. This way a structure can be maintained in multiple trees 
at the same moment and can put in the tree without allocating node structures 
(useful when the memory manager is not yet functional).

However, the AVL tree must maintan order and compare elements in order to achieve that.
The user must provide a \texttt{compare} function and a \texttt{match} function.

\begin{description}
\item[compare]
  The \texttt{compare} function accepts two nodes and returns a value lesser than, equal or greater
  than zero if the first argument is respectively lesser than, equal or greater than the second
  argument.
  
  This function is called when a new node is inserted in the tree.
  
\item[match]
  The \texttt{compare} function could be used also for finding an element but
  the user should create a dummy object with the specified key. However tha main
  reason to implement a separate \texttt{match} function is to implement \emph{range searching}
  \footnote{Used expecially in memory management.}.
\end{description}

In order to simplify the syntax of the \texttt{avl\_add} and \texttt{avl\_find}
macros the \texttt{compare} and \texttt{match} functions must be named \texttt{typename\_avlcompare}
and \texttt{typename\_avlmatch} respectively. This naming implies that the type name 
should not contain spaces\footnote{\textsf{struct}s must be typedefed.\acaros\ coding conventions
already guarantee this.}.

